\chapter{نتایج تجربی}\label{chap:chap4}

در این بخش به نتایج تجربی حاصل از پیاده‌سازی روش‌های معرفی شده در فصل
\ref{chap:chap3}
خواهیم پرداخت.
%%%%%%%%%%%%%%%%%%%%%%%%%%%%%%%%%%%%%%%%%%%%%%
%    4.1
%%%%%%%%%%%%%%%%%%%%%%%%%%%%%%%%%%%%%%%%%%%%%%
\section{معیار ارزیابی}
فرض کنید که داده‌های آزمایشی ما به صورت زوج مرتب‌های 
$ \Big\{ (x_i, y_i) \Big\}_{i=1}^{N_{test}} $
موجود باشد. که در آن $x_i$ تصویر، $y_i$ برچسب مربوط به آن و $N_{test}$ تعداد تصاویر آزمایشی است. همچنین در پایگاه داده پرندگان کلتک $N_{test}$ برابر با ۵۷۹۴ است.

حال برای ارزیابی تابع دسته‌بند یعنی
$ f(x) $
از فرمول

\begin{equation}
	mA = \frac{1}{\mathbb{C}} \sum_{c=1}^{\mathbb{C}} \frac{1}{|\alpha(c)|} \sum_{i \in \alpha(c)} \mathbb{I}(f(x_i) = y_i)
	\label{eq:4:ma}
\end{equation}
دقت میانگین را محاسبه می‌کنیم.


\section{نتایج نهایی روش پیشنهادی} \label{chap:4:final}

حالت دوم موسوم به DeepRF(All) که در آن روش پیشنهاد شده به غیر از مستطیل محیطی سر و بدن پرنده، لازم است که مستطیل محیطی کل پرنده را نیز تخمین بزند. به دلیل استفاده از جنگل تصادفی در این روش و ذات تصادفی بودن آن، آزمایش‌ها را سه بار انجام داده‌ایم و میانگین و انحراف معیار دقت را گزارش خواهیم کرد.

\begin{table}
	\centering
	\caption{دقت روش‌های نهایی پیشنهادی (مشخص شده توسط *) در مقایسه با روش مرز دانش. برای روش‌های پیشنهادی میانگین و انحراف معیار در سه آزمایش گزارش شده است.}
	\label{tbl:4:f_res}
	
	\footnotesize{
		\begin{tabular}{|c|c|c|c|c|c|C{2cm}|}
			\cline{3-6}
			\multicolumn{1}{r}{ }	&						&  \multicolumn{2}{c|}{آموزش} & \multicolumn{2}{c|}{آزمایش} \\
			\hline 	سال		&	نام					&	پنجره محیطی	&	مکان اجزا 	&	پنجره محیطی	&	مکان اجزا & 		دقت میانگین دسته‌بندی (درصد)	\\
			\hline 
			\hline 	2014 	& 	PRCNN \cite{partrcnn}	& \checkmark	& \checkmark	& \checkmark	& 			& 	76٫37				\\
			\hline 	- 		& 	DeepRF *				& \checkmark 	& \checkmark	&  \checkmark	& 			& 	73٫78 ($\pm$0٫32)		\\
			\hline
			\hline 	2014 	& 	PRCNN \cite{partrcnn}	& \checkmark	& \checkmark	& 			& 			& 	73٫89				\\
			\hline 	- 		& 	DeepRF(All) *			& \checkmark 	& \checkmark	& 			& 			& 	72٫02 ($\pm$0٫33)		\\
			\hline
		\end{tabular} 
	}
\end{table}

همانطور که در جدول
\ref{tbl:4:f_res}
نیز ذکر شده است، روش پیشنهادی موسوم به DeepRF به دقت میانگین 73٫78 دست پیدا می‌کند که قابل مقایسه با روش مرز دانش با دقت میانگین 76٫37 است. همچنین روش پیشنهادی موسوم به DeepRF(All) به دقت میانگین 72٫02 دست پیدا می‌کند. روش مرز دانش نیز به دقت میانگین 73٫89 درصد در حالتی که مستطیل محیطی در زمان آزمایش در دسترس نیست دست پیدا می‌کند.

به طور کلی می‌توان گفت که روش پیشنهاد شده از نظر کارایی بسیار سریع‌تر از روش مرز دانش است، از نظر سادگی بسیار ساده‌تر از آن است و از نظر کاربردی در موارد مختلف دیگری نیز با کمترین تغییرات می‌تواند مورد استفاده قرار بگیرد. با این حال از نظر دقت میانگین دسته‌بندی نتیجه‌ای بسیار نزدیک نسبت به آن کسب می‌کند.
